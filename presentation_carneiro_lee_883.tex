% presentation_carneiro_lee_883.tex
% 20-30 minute presentation on Carneiro and Lee (2009)
% Jackson Bunting; Drew Vollmer 03-25-2017

\documentclass{beamer}

\usepackage{amsfonts} % For \mathbb command
\usepackage{amsmath} % for align
\usepackage{graphicx} % for graphics


\allowdisplaybreaks % so that aligned equations can be broken across pages
\newcommand{\E}{\mathrm{E}} % Expectation operator
\newcommand{\Var}{\mathrm{Var}} % Variance operator
\mathchardef\mhyphen="2D % Create a hyphen for math mode
\newcommand*\diff{\mathop{}\!d} % nicely formatted integral dx

\usetheme{madrid}

\begin{document}



\title[Distributions of Potential Outcomes]{Estimating Distributions
  of Potential Outcomes Using Local Instrumental Variables}
\subtitle{Carneiro and Lee, Journal of Econometrics (2009)}
\author[]{Jackson Bunting and Drew Vollmer}
\frame{\maketitle}


\begin{frame}{Overview}

% Nontechnical overview of big ideas

\end{frame}


\begin{frame}{Econometric Model} % Or framework, whichever is appropriate

  % Show the outcome model and state that it is the same as the one in
  % our MTE lecture.

  % Mention a few assumptions, but only the ones that we will directly
  % invoke (refer back to MTE lecture for others)

\end{frame}


% Context for theorem 1: essential background on the MTE and local
% instrumental variables to put the result in context
\begin{frame}{Identification Context}
\end{frame}


% Shrink to make the math type just a bit smaller
\begin{frame}[shrink = 1]{Key Identification Result}

\begin{theorem}
  Under the potential outcomes framework, selection equation, and
  technical assumptions,

\vspace{-.25cm}
\begin{align*}
  \E\left[ G(Y_1) | X = x, U_D = p \right] = &\E \left[ G(Y) | X = x, \mu_D(Z) = p, D = 1 \right] \\
  &+ p \frac{\partial \E\left[ G(Y) | X = x, \mu_D(Z) = p, S = 1 \right]}{\partial p} \\
  \E\left[ G(Y_0) | X = x, U_D = p \right] = &\E \left[ G(Y) | X = x, \mu_D(Z) = p, D = 0 \right] \\
  &- (1 - p) \frac{\partial \E\left[ G(Y) | X = x, \mu_D(Z) = p, S = 0 \right]}{\partial p} \\
\end{align*}
\end{theorem}

\end{frame}


% Need text a bit smaller, so use the shrink parameter
\begin{frame}[shrink = 10]{Proof of Theorem}

\begin{proof}
\[
\begin{aligned}
\action<+->{\E [ G(Y) | &X = x, \mu_D(Z) = p, D = 1 ] = \E \left[ G(Y) | X = x, \mu_D(Z) = p, U_D < p \right] \\}
\action<+->{&= \E \left[ G(Y_1) | X = x, \mu_D(Z) = p, U_D < p \right] = \E \left[ G(Y_1) | X = x, U_D < p \right] \\}
\action<+->{&= \frac{1}{\Pr(U_D < p)} \int_0^p \E \left[ G(Y_1) | X = x, U_D = u_D \right] f_{U_D | X}(u_D | x) \diff u_D \\}
\action<+->{&= \frac{1}{p} \int_0^p \E \left[ G(Y_1) | X = x, U_D = u_D \right] \int \underbrace{f_{U_D | X, Z}(u_D | x, z)}_{1} f_{Z|X}(z | x) \diff z \diff u_D \\}
\action<+->{&= \frac{1}{p} \int_0^p \E \left[ G(Y_1) | X = x, U_D = u_D \right] \diff u_D \\}
\end{aligned}
\]

\action<+->{Conclude by differentiating both sides with respect to $p$.}

\end{proof}
\end{frame}


% Explain why the theorem matters
\begin{frame}{Implications}
\end{frame}


\end{document}
