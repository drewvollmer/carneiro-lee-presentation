% presentation_carneiro_lee_883.tex
% 20-30 minute presentation on Carneiro and Lee (2009)
% Jackson Bunting; Drew Vollmer 03-25-2017

\documentclass{beamer}

\usepackage{amsfonts} % For \mathbb command
\usepackage{amsmath} % for align
\usepackage{graphicx} % for graphics


\allowdisplaybreaks % so that aligned equations can be broken across pages
\newcommand{\E}{\mathrm{E}} % Expectation operator
\newcommand{\Var}{\mathrm{Var}} % Variance operator
\mathchardef\mhyphen="2D % Create a hyphen for math mode
\newcommand*\diff{\mathop{}\!d} % nicely formatted integral dx

\usetheme{madrid}

\begin{document}



\title[Distributions of Potential Outcomes]{Estimating Distributions
  of Potential Outcomes Using Local Instrumental Variables}
\subtitle{Carneiro and Lee, Journal of Econometrics (2009)}
\author[]{Jackson Bunting and Drew Vollmer}
\frame{\maketitle}


\begin{frame}{Overview}

% Nontechnical overview of big ideas
\begin{itemize}

\item Context: we know that the marginal treatment effect (MTE) can be
  used to construct other treatment effects of interest
\begin{itemize}
\item MTE can be estimated using local instrumental variables
\end{itemize}

\pause

\item Problem: simple treatment effects, like $\E(Y_1 - Y_0 | \dots)$,
  might be too crude to properly assess policies
\begin{itemize}
\item Consider returns to college, which vary across the population

\item Policy evaluation should care about the marginal enrollee, but
  the marginal effect depends on who is already enrolled
\end{itemize}

\pause

\item Satisfying answers require a way to estimate the distribution of
  changes in potential outcomes
\begin{itemize}
\item This is the goal of Carneiro and Lee's paper
\end{itemize}

\end{itemize}

\end{frame}




\begin{frame}{Econometric Model} % Or framework, whichever is appropriate

\begin{itemize}

\item Setting is almost identical to the one for MTEs

\item $Y_i = \mu_i(X, U_i)$ for $i \in \{0, 1\}$

\item Selection follows $D = \mathbb{I}[ \mu_D(Z) \geq U_D ]$

\end{itemize}

\pause

\begin{enumerate}

\item $\mu_D(Z)$ is nondegenerate conditional on $X$

\item $(U_i, U_D) \perp Z | X$

\item Distribution of $U_D$ is abs. continuous $\dots$

\item Normalize unobservables so $U_i | X, Z \sim U[0, 1]$

\pause

\item $\E | G(Y_i) | < \infty$

\end{enumerate}

\end{frame}


% Context for theorem 1: essential background on the MTE and local
% instrumental variables to put the result in context
\begin{frame}{Identification Context}

\begin{itemize}

% How does P(Z) connect to our notation?
\item Local IV result:

\begin{align*}
  \frac{\partial \E \left( Y | X = x, P(Z) = p \right) }{\partial p} &= \Delta^{MTE}(x, p) \\
  &= \E \left( Y_1 - Y_0 | X = x, U_d = u_d \right)
\end{align*}

% Comments on implications and estimation
\pause

\item Carneiro and Lee's contribution: generalize estimation of the
  MTE to an arbitrary function $G(Y)$

\end{itemize}

\end{frame}





% Shrink to make the math type just a bit smaller
\begin{frame}[shrink = 1]{Key Identification Result}

\begin{theorem}
  Under the potential outcomes framework, selection equation, and
  technical assumptions,

\vspace{-.25cm}
\begin{align*}
  \E\left[ G(Y_1) | X = x, U_D = p \right] = &\E \left[ G(Y) | X = x, \mu_D(Z) = p, D = 1 \right] \\
  &+ p \frac{\partial \E\left[ G(Y) | X = x, \mu_D(Z) = p, S = 1 \right]}{\partial p} \\
  \E\left[ G(Y_0) | X = x, U_D = p \right] = &\E \left[ G(Y) | X = x, \mu_D(Z) = p, D = 0 \right] \\
  &- (1 - p) \frac{\partial \E\left[ G(Y) | X = x, \mu_D(Z) = p, S = 0 \right]}{\partial p} \\
\end{align*}
\end{theorem}

\end{frame}


% Need text a bit smaller, so use the shrink parameter
% Tutorial on using action and others in beamer: https://www.sharelatex.com/blog/2013/08/20/beamer-series-pt4.html
\begin{frame}[shrink = 10]{Proof of Theorem}

\begin{proof}
\[
\begin{aligned}
\action<1->{\E [ G(Y) | &X = x, \mu_D(Z) = p, D = 1 ] = \E \left[ G(Y) | X = x, \mu_D(Z) = p, U_D < p \right] \\}
\action<1->{&= \E \left[ G(Y_1) | X = x, \mu_D(Z) = p, U_D < p \right] = \E \left[ G(Y_1) | X = x, U_D < p \right] \\}
\action<2->{&= \frac{1}{\Pr(U_D < p)} \int_0^p \E \left[ G(Y_1) | X = x, U_D = u_D \right] f_{U_D | X}(u_D | x) \diff u_D \\}
\action<3->{&= \frac{1}{p} \int_0^p \E \left[ G(Y_1) | X = x, U_D = u_D \right] \int \underbrace{f_{U_D | X, Z}(u_D | x, z)}_{1} f_{Z|X}(z | x) \diff z \diff u_D \\}
\action<3->{&= \frac{1}{p} \int_0^p \E \left[ G(Y_1) | X = x, U_D = u_D \right] \diff u_D \\}
\end{aligned}
\]

\action<3->{Conclude by differentiating both sides with respect to $p$.}

\end{proof}
\end{frame}


% Explain why the theorem matters
\begin{frame}{Implications}
\end{frame}


% Construction of treatment effects from theorem 1
\begin{frame}{Getting Treatment Effects}
\end{frame}


\end{document}
